\documentclass{beamer}

\usetheme{PSU}
\usepackage{psu-colors}
\usepackage{psu-logos}

\usepackage{pgfplots, pgfplotstable}
\usepackage{amsmath, amssymb, amsthm, amsfonts}

\pgfplotsset{compat=newest}

\setmainfont{Acumin Pro}[
    UprightFont=*-Regular,
    BoldFont=*-Bold,
    ItalicFont=*-Italic,
]

\title{Organ tuning consulting}
\subtitle{Team C -- CADES Lab}
\author[Cyanam, Nguyen, Pinochet-Soto]{Meghana Cyanam \and Hai Nguyen \and Gabriel Pinochet-Soto}
\institute{Portland State University}
\date{\today}

\begin{document}
    {
    \setbeamercolor{background canvas}{bg=psuForestGreen}
    \begin{frame}
		\titlepage
	\end{frame}
    }

    \begin{frame}
        \frametitle{Outline}
        \tableofcontents
    \end{frame}

    \section{Introduction}
    \subsection{What is organ tuning?}
	\begin{frame}
        \frametitle{Introduction: What is organ tuning?}
		\begin{itemize}
            \item<1->
                Pipe organs have been around for hundreds of years, and making them sound just
                right is a special job called intonation, or \emph{voicing}.

			\item<2->
                The way a pipe looks and is built -- like its size, shape, and material-- can
                change how it sounds.

			\item<3->
                These details are decided first, and then the pipes are fine-tuned before being
                added to the organ.
		\end{itemize}
	\end{frame}

% isBourdon,flueDepth,frequency,cutUpHeight,diameterToe,acousticIntensity,partial1,partial2,partial3,partial4,partial5,partial6,partial7,partial8
% 0,1.10E-03,110.00,1.70E-02,8.00E-03,68.00,99.00,97.00,97.00,81.00,64.00,45.00,30.00,12.00 is Good
% 0,1.10E-03,174.60,1.00E-02,1.10E-02,111.00,34.00,99.00,17.00,11.00,59.00,28.00,0.00,32.00 is Bad
    \subsection{Good vs. bad voicing}
    \begin{frame}
        \frametitle{Introduction: Good vs. bad voicing}
        \begin{columns}
            \begin{column}{0.5\textwidth}
                \centering
                \begin{figure}[htbp]
                    \begin{tikzpicture}
                        \begin{axis}[
                            width=0.9\textwidth,
                            ybar,
                            xlabel={Partial},
                            ylabel={Relative amplitude},
                            symbolic x coords={1st, 2nd, 3rd, 4th, 5th, 6th, 7th, 8th},
                            ymin=0, ymax=100,
                            ytick={0, 50, 100},
                            tick label style={font=\large},
                            label style={font=\small},
                        ]
                          \addplot [
                            color=psuOrange,
                            fill=psuOrange,
                            opacity=0.75,
                        ] coordinates {(1st,99) (2nd,97) (3rd,97) (4th,81) (5th,64) (6th,45) (7th,30) (8th,12)};
                        \end{axis}
                    \end{tikzpicture}
                    \caption{\textbf{Good voicing}: Good distribution of harmonics.}
                    \label{fig:good_voicing}
                \end{figure}
            \end{column}
            \begin{column}{0.5\textwidth}
                \centering
                \begin{figure}
                    \begin{tikzpicture}
                        \begin{axis}[
                            width=0.9\textwidth,
                            ybar,
                            xlabel={Partial},
                            ylabel={Relative amplitude},
                            symbolic x coords={1st, 2nd, 3rd, 4th, 5th, 6th, 7th, 8th},
                            ymin=0, ymax=100,
                            ytick={0, 50, 100},
                            tick label style={font=\large},
                            label style={font=\small},
                        ]
                            \addplot [
                                color=psuOrange,
                                fill=psuOrange,
                                opacity=0.75,
                            ] coordinates {(1st,34) (2nd,99) (3rd,17) (4th,11) (5th,59) (6th,28) (7th,0) (8th,32)};
                        \end{axis}
                        \end{tikzpicture}
                    \caption{\textbf{Bad voicing}: Bad distribution of harmonics.}
                    \label{fig:bad_voicing}
                \end{figure}
            \end{column}
        \end{columns}
    \end{frame}

    \subsection{Ising number}
    \begin{frame}
        \frametitle{Introduction: Ising number}
        \begin{alertblock}{Ising number}
            \begin{equation}
                \label{eq:ising}
                \mathsf{I}
                =
                \frac{v}{\omega}\sqrt{\frac{d}{h^3}}
                =
                \frac{1}{\omega}\sqrt{\frac{2 P d}{\rho h^3}},
            \end{equation}
        \end{alertblock}
        \begin{enumerate}
            \item Can the Ising formula~\ref{eq:ising} be improved, refined, or
                corrected, in order to match measured data?
            \item Can the flow rate be included in the Ising formula~\ref{eq:ising}?
            \item Can we obtain a model reliable enough, that can be used to dimension the
                pipes before the voicing stage?
        \end{enumerate}
    \end{frame}

    \section{Action Plan}
    \subsection{Methodology}
    \begin{frame}
        \frametitle{Action Plan}
        \begin{enumerate}
            \item<1-> Semianalytical methods
            \item<2-> FEM simulations
            \item<3-> Machine learning (DNN) on JAX
            \item<4-> Generalized Additive Models (GAM)
            \item<5-> \emph{We are open to ideas!}
        \end{enumerate}
    \end{frame}
\end{document}
