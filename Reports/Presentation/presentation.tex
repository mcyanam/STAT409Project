\documentclass{beamer}

\usetheme{PSU}
\usepackage{psu-colors}
\usepackage{psu-logos}

\setmainfont{Acumin Pro}[
    UprightFont=*-Regular,
    BoldFont=*-Bold,
    ItalicFont=*-Italic,
]

\title{Organ tuning consulting}
\subtitle{Team C -- CADES Lab}
\author[Cyanam, Nguyen, Pinochet-Soto]{Meghana Cyanam \and Hai Nguyen \and Gabriel Pinochet-Soto}
\institute{Portland State University}
\date{\today}

\begin{document}
    {
    \setbeamercolor{background canvas}{bg=psuForestGreen}
    \begin{frame}
		\titlepage
	\end{frame}
    }

    \begin{frame}
        \frametitle{Outline}
        \tableofcontents
    \end{frame}

    \section{Introduction}
    \subsection{What is organ tuning?}
	\begin{frame}
        \frametitle{Introduction: What is organ tuning?}
		\begin{itemize}
            \item<1->
                Pipe organs have been around for hundreds of years, and making them sound just
                right is a special job called intonation, or \emph{voicing}.

			\item<2->
                The way a pipe looks and is built -- like its size, shape, and material-- can
                change how it sounds.

			\item<3->
                These details are decided first, and then the pipes are fine-tuned before being
                added to the organ.
		\end{itemize}
	\end{frame}

    \subsection{Ising number}
    \begin{frame}
        \frametitle{Introduction: Ising number}
        \begin{alertblock}{Ising number}
            \begin{equation}
                \label{eq:ising}
                \mathsf{I}
                =
                \frac{v}{\omega}\sqrt{\frac{d}{h^3}}
                =
                \frac{1}{\omega}\sqrt{\frac{2 P d}{\rho h^3}},
            \end{equation}
        \end{alertblock}
        \begin{enumerate}
            \item Can the Ising formula~\ref{eq:ising} be improved, refined, or
                corrected, in order to match measured data?
            \item Can the flow rate be included in the Ising formula~\ref{eq:ising}?
            \item Can we obtain a model reliable enough, that can be used to dimension the
                pipes before the voicing stage?
        \end{enumerate}
    \end{frame}

    \section{Action Plan}
    \subsection{Methodology}
    \begin{frame}
        \frametitle{Action Plan}
        \begin{enumerate}
            \item<1-> Semianalytical methods
            \item<2-> FEM simulations
            \item<3-> Machine learning (DNN) on JAX
            \item<4-> Generalized Additive Models (GAM)
            \item<5-> \emph{We are open to ideas!}
        \end{enumerate}
    \end{frame}
\end{document}
